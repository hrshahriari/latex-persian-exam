% A Sample exam in Persian language
% By Hamid Reza Shahriari (linkedin.com/in/hrshahriari)

% You can find more details about exam package here:
%		https://www.overleaf.com/learn/latex/Typesetting_exams_in_LaTeX
%		https://www.nic.funet.fi/pub/TeX/CTAN/macros/latex/contrib/exam/examdoc.pdf

%\documentclass[answers]{exam}		% Use to show answers
\documentclass{exam}
\pointname{نمره}
\boxedpoints

\usepackage{amsmath}
\usepackage{amssymb}
\usepackage{amsmath}
\usepackage{listings}		% for listing SQL code
\lstset{language=SQL, basicstyle=\sffamily, stringstyle=\ttfamily, columns=fullflexible}  % Change it for other programming languages

\usepackage{booktabs}

\usepackage{fancybox}
\usepackage{color}
\usepackage{relsize}
\usepackage{newlfont}

%\renewcommand{\solutiontitle}{\hfill\textbf{\rl{ پاسخ:}}\par}
%\SolutionEmphasis{}
%\SolutionEmphasis{\itshape\Large}

%\printanswers		% Force to show answers

\shadedsolutions
\definecolor{SolutionColor}{rgb}{0.8,0.9,1}

% A command to hide a block or commnet out
\newcommand{\hide}[1]{}

\usepackage{xepersian}
%\settextfont{XB Niloofar}
\settextfont{Calibri}		% Persian text font

\newcommand{\thecourse}{اصول طراحی پایگاه داده}  % Specify the course title
\newcommand{\examtitle}{آزمونک اول}  					% Specify the exam title
\newcommand{\theteacher}{حمیدرضا شهریاری} 		% Specify the teacher name
\newcommand{\examtime}{60 دقیقه} 						% Specify the exam time limit

\newcommand{\lsf}[1]{\lr{\textsf{#1}}}					% Use \lsf{english text} for English text in letf-to-right and serif font family
\newcommand{\TF}{\dotfill  $\bigcirc$ درست $\bigcirc$ نادرست}	% True/False questions

\firstpageheader{}{بسمه تعالی}{}

\setlength{\topmargin}{-2cm}
\setlength{\textheight}{26cm}

\renewcommand{\baselinestretch}{1.3}

\runningheadrule
\runningheader{\thecourse}{}{\examtitle}
\footer{\iflastpage{انتهای آزمون} }
{صفحه \thepage\ از \numpages}{}

\SolutionEmphasis{\itshape}
%\printanswers

\shadedsolutions
\definecolor{SolutionColor}{rgb}{0.8,0.9,1}

\begin{document}
			
\shadowbox{
			\begin{minipage}{\linewidth}
			\centering
						    {\huge \thecourse \\}
						     \examtitle - زمان: \examtime \\
						    مدرس:  \theteacher \\
						    \today
			\end{minipage}		
}

\makebox[0.5\textwidth]{نام:\enspace\dotfill} 
\makebox[0.5\textwidth]{شماره دانشجویی:\enspace\dotfill}

\pointname{نمره}
\hqword{پرسش}
\hpword{بارم}
\hsword{نمره}
\htword{جمع}

\addpoints
\begin{center}
\gradetable[h][questions]
\end{center}

\textbf{توجه:}
\begin{itemize}
\item آزمون کتاب و جزوه بسته است.
\item	استفاده از هر گونه وسایل محاسباتی و  ارتباطی (مانند لپ تاپ، موبایل، تبلت و غیره) ممنوع است.
%\item ابتدا همه سوالها را مرور نمایید و اگر ابهامی در سوال وجود دارد، در همان 10 دقیقه اول بپرسید.
\item سوالها را با همان فرضیات داده شده حل کنید. در صورتی که فرض جدیدی لازم است، صریحاً بنویسید.
\end{itemize}

\rule{\textwidth}{2pt}

\begin{questions}

\question[4]
هر یک از موارد زیر کوتاه و دقیق (فارسی یا انگلیسی) پاسخ دهید:
\begin{parts}
\part  مجموعه صفاتی که به طور منحصر به فرد یک تاپل را مشخص می‌کنند.
\answerline

\part توصیف انتزاعی از داده‌ها
\answerline

\end{parts}

\question[2]
درستی یا نادرستی هر یک از عبارتهای زیر را تعیین کنید :
\begin{parts}
\part  در هر رابطه حداکثر یک سوپر کلید وجود دارد. \TF
\part	کلید اصلی یک سوپرکلید است. \TF
\end{parts}


\question[2]
هر یک از موارد زیر کوتاه و دقیق (فارسی یا انگلیسی) پاسخ دهید: (روش دیگر برای جای پاسخ)
\begin{parts}
\part  مجموعه صفاتی که به طور منحصر به فرد یک تاپل را مشخص می‌کنند.
\fillin

\part توصیف انتزاعی از داده‌ها
\fillin

\end{parts}

\rule{\textwidth}{1pt}	 % Draw a line

\question[22]
{\bf
برای پرسشهای زیر، گزینه‌ درست را در پاسخ‌نامه انتخاب کنید.
}

\begin{parts}
\renewcommand{\thepartno}{\arabic{partno}.\thequestion}
\part
کدام گزینه درست است؟
\begin{choices}
\choice	گزینه اول
\choice گزینه دوم
\choice 	گزینه سوم
\choice 	گزینه چهارم
\end{choices}

\part
سوال چند گزینه‌ای که پاسخها در یک خط هستند. \\
\begin{oneparchoices}
\choice اول	\choice دوم	\choice سوم	\choice چهارم
\end{oneparchoices}

\end{parts}

\newpage

\question
شمای پایگاه داده‌های زیر که در آن کلیدهای اصلی دارای زیرخط هستند را در نظر بگیرید که \lsf{suppliers} جدول تامین کنندگان، \lsf{parts} جدول قطعات، و \lsf{catalogue} جدول کاتالوگ قطعات که چه تامین کننده ای چه قطعه ای را با چه قیمتی تامین می‌کند:
\begin{latin}
\begin{sffamily}

suppliers (\underline{sid}, sname, saddress) \\
parts (\underline{pid}, pname, pcolor)  \\
catalogue (\underline{sid}, \underline{pid}, cost) 

\end{sffamily}
\end{latin}
برای هر یک از موارد زیر یک عبارت پرس‌وجو در زبان \lr{SQL} بنویسید:
\begin{parts}
\part[10]
اسامی تامین کنندگانی که قطعه به شماره \textsf{p1} را تأمین می‌کنند.
\begin{latin}

\begin{solutionorbox}[\stretch{1}]
\begin{lstlisting}
select  sname
from suppliers as S, catalogue as C
where   S.sid = C.sid and  
               C.pid = 'p1';

select  sname
from suppliers as S natural join  catalogue as C
where   C.pid = 'p1';

\end{lstlisting}
\end{solutionorbox}
\end{latin}


\part[10]
اسامی آن دسته از تأمین کنندگانی را که قطعه‌ای با رنگ قرمز تأمین می‌کنند.

\begin{latin}

\begin{solutionorbox}[\stretch{1}]

\begin{lstlisting}
select sname
from suppliers as S, catalogue as C, parts as P
where 
	S.sid = C.sid and
	C.pid = P.pid and
	P.pcolor = 'red';

\end{lstlisting}
\end{solutionorbox}
\end{latin}


\part[10]
اسامی تأمین کنندگانی که ارزانترین قیمت را برای قطعه شماره \textsf{p1} ارائه می دهند.

\begin{latin}

\begin{solutionorbox}[\stretch{1}]

\begin{lstlisting}
Select sname
from suppliers as S, catalogue as C, 
where  S.sid = C.sid
    	  and C.pid = 'p1'
          and C.cost = (
              		select MIN(cost)
              		from Catalogue  as T
               		where T.pid = "p1");

\end{lstlisting}
\end{solutionorbox}
\end{latin}


\newpage


\part[10]
اسامی آن دسته از قطعاتی که توسط همه تأمین کنندگان تأمین می‌شوند.

راهنمایی: با توجه به این که در SQL  سور عمومی یا \lr{(For all) $\forall$}  مستقیم پشتیبانی نمی شود، باید از معادل آن به صورت دو بار منفی سور وجودی استفاده کرد.
به عبارت دیگر پرس و جوی بالا را به این صورت معادل می نویسیم:

هر قطعه‌ای که تأمین کننده‌ای وجود ندارد که آن قطعه را تأمین نکند.
 
% $\forall pid \in Parts, \nexists sid \in Suppliers, (sid, pid) \notin Catalogue $

\begin{latin}

\begin{solutionorbox}[\stretch{1}]

در SQL معادل عبارت بالا را به هر دو صورت زیر می‌توان نوشت:
$\forall pid \in Parts, \nexists sid \in Suppliers, (sid, pid) \notin Catalogue $
\begin{lstlisting}
select distinct P.pname
from parts as P
where not exists 
	(select *
  	from suppliers as S 
  	where not exists
  			(select *
  	 		from catalogue as C
	 		where (S.sid, P.pid) = (C.sid, C.pid));

select distinct P.pname
from parts as P
where not exists 
	(select *
  	from suppliers as S 
  	where (S.sid, P.pid) not in
  			(select sid, pid
  	 		from catalogue);

\end{lstlisting}
\end{solutionorbox}
\end{latin}

\newpage

\part[10]
اسامی تمامی تأمین کنندگان شهر تبریز که قطعاتی با رنگ قرمز یا سبز تأمین می‌کنند.


\begin{latin}

\begin{solutionorbox}[\stretch{1}]
\begin{lstlisting}
select sname
from suppliers as S, catalogue as C, parts as P
where 
	S.sid = C.sid and
	S.saddress = 'Tabriz' and
	C.pid = P.pid and
	(P.pcolor = 'red' or P.pcolor='green');

\end{lstlisting}

\end{solutionorbox}
\end{latin}
%\newpage
\part[10]
اسامی همه تأمین‌کنندگانی که فقط قطعات با رنگ آبی را تأمین می‌کنند.
راهنمایی: به این صورت معادل در نظر بگیرید: همه تأمین کنندگانی که قطعه غیر آبی تأمین نمی‌کنند.
\begin{latin}

\begin{solutionorbox}[\stretch{1}]
\begin{lstlisting}
select sname
from suppliers as S
where not exists 
	(select *
	from catalogue as C, parts as P
	S.sid = C.sid and
	C.pid = P.pid and
	P.pcolor  != 'blue');

\end{lstlisting}
\end{solutionorbox}
\end{latin}



\part[10]
قیمت همه قطعات که تأمین کننده آنها در شهر تهران است را 10 درصد افزایش دهید.
\begin{latin}

\begin{solutionorbox}[\stretch{1}]
\begin{lstlisting}

update Catalogue
set cost = cost * 1.1
where sid in 
        (select sid
		from Suppliers
		where saddress = 'Tehran');

\end{lstlisting}

\end{solutionorbox}
\end{latin}



\end{parts}

\hide{

\question
برای پایگاه داده پرسش قبل، پرس و جوهای خواسته شده را با حساب رابطه ای (تاپلی یا دامنه‌ای) بنویسید.
\begin{parts}
\part[10]
اسامی آن دسته از قطعاتی که توسط تمامی تأمین کنندگان تأمین می‌شوند (حساب رابطه ای).
\makeemptybox{\stretch{1}}

\part[10]
اسامی آن دسته از تأمین کنندگانی را که قطعات با رنگ قرمز تأمین می‌کنند (حساب رابطه ای).
\makeemptybox{\stretch{1}}

\end{parts}

\question
شمای پایگاه داده‌های زیر که در آن کلیدهای اصلی دارای زیرخط هستند را در نظر بگیرید:  
\begin{latin}
\begin{sffamily}

employee (\underline{employee\_name}, street, city)\\
works (\underline{employee\_name}, \underline{company\_name}, salary)\\
company (\underline{company\_name}, city)\\
manages (\underline{employee\_name}, manager\_name)

\end{sffamily}
\end{latin}
برای هر یک از موارد زیر یک عبارت پرس‌وجو در زبان \lr{SQL} بنویسید:
\begin{parts}
%\part[10]
%دستورات \lr{SQL} لازم برای ایجاد این پایگاه داده‌ها را بنویسید.
%\makeemptybox{\stretch{1}}
\part[10]
پایگاه داده را به گونه‌ای ویرایش کنید که تغییر مکان آقای رضایی به شهر ارومیه را  منعکس کند.
\makeemptybox{\stretch{1}}
\newpage
\part[10]
از حقوق تمامی مدیران شرکت {رصد} به میزان 10\% کم کنید.
\makeemptybox{\stretch{1}}

\part[10]
اسامی تمامی کارمندانی را لیست کنید که در همان شهر و خیابانی که مدیر آن‌ها در آن ساکن است، ساکن هستند.
\makeemptybox{\stretch{1}}

\part[10]
اسامی تمامی کارمندانی را که حقوقی بیشتر از میانگین حقوق تمامی کارمندان شرکتشان دریافت می‌کنند.
\makeemptybox{\stretch{1}}
\part[10]
اسامی مدیرانی که کمترین حقوق را در بین سایر مدیران دریافت می‌کنند.
\makeemptybox{\stretch{1}}
\end{parts}
} %hide

\end{questions}

\end{document}
